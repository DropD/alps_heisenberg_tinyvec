\documentclass[a4paper]{scrartcl}

\usepackage{mathtools}
\usepackage{cleveref}
\usepackage{graphicx}
\usepackage{caption}
\usepackage{subcaption}
\usepackage{listings}
\usepackage[usenames,dvipsnames,svgnames]{xcolor}
\lstset{
    language=C++,
    basicstyle=\footnotesize\ttfamily,
    breaklines=true,
    keywordstyle=\color{DarkGreen}
    }
\DeclarePairedDelimiter\floor{\lfloor}{\rfloor}
\newcommand{\sse}{SSE}

\author{Rico Häuselmann}
\title{highlevel optimization of tinyvector}

\begin{document}
\section*{Introduction}
The \texttt{tinyvector} class template is intended to be used inside the inner loops of physics programs such as monte carlo spin simulations,
inside the ALPS framework. Originally developed to hold spin vectors for the $O(n)$-Model it is designed for a very small amount of dimensions only
as the runtime for such simulations scales exponentially in the number of dimensions.

Typically in the inner loop of the targeted kind of simulation not very complex operations are used on spin vectors; mostly addition, multiplication
and dot products are applied. Thus, the four basic operations ($+, -, *, /$) were optimized using intrinsic vector instructions. 
The optimized operations were benchmarked against the naive for loop implementation to see whether a speedup was obtained.

\section*{Methods}
\subsection*{Optimization}
One of the \texttt{tinyvector}s template parameters, \texttt{Opt}, is used as a tag to determine wether optimization is desired by the user.
The following operators are overloaded for \texttt{tinyvector<double, N, INTRIN\_OPT>} to use either AVX, SSE or simple loop unrolling, choosing the highest 
level of vectorization available among them.

\begin{lstlisting}
template <int N>
inline const tinyvector<double, N, INTRIN_OPT> & operator+=
template <int N> 
inline const tinyvector<double, N, INTRIN_OPT> & operator-=
template <int N> 
inline const tinyvector<double, N, INTRIN_OPT> & operator*=
template <int N> 
inline const tinyvector<double, N, INTRIN_OPT> & operator/=
\end{lstlisting}

The process of dispatching and chaining the calls to the most vectorized version of the basic operation for doubles will now be described using
\texttt{operator+=} as an example.

First the maximum amount of doubles on which will be operated simultaneously is stored in \texttt{STEP}.
\begin{lstlisting}                                                                                                    
/** 
 * determine the vectorization step length, aka how many doubles can we fit into a register? 
 */
#ifdef __AVX__
    const int STEP = 4;
#else
    #ifdef __SSE2__
        const int STEP = 2;
    #else
        const int STEP = 1;
    #endif
#endif                                                                                                                  
\end{lstlisting}

The Operator struct \texttt{\_plus} defines \texttt{apply} functions for single values, vectors, SSE and AVX registers.
Each just adds the \texttt{right} to the \texttt{left} argument in place.

\begin{lstlisting}
/**
 * operator kernels for different vectorization levels
 */
struct _plus
{
    template <class T>
    static inline const void apply(T & left, const T & right) {
        left += right;
    }
    template <class vec>
    static inline const void apply(vec & left, const vec & right, const int start) {
        left[start] += right[start];
    }
#ifdef __SSE2__
    static inline const __m128d apply(__m128d & mml, __m128d & mmr) {
        return _mm_add_pd(mml, mmr);
    }
#endif
#ifdef __AVX__
    static inline const __m256d apply(__m256d & mml, __m256d & mmr) {
        return _mm256_add_pd(mml, mmr);
    }
#endif
};
\end{lstlisting}

The \texttt{\_vectorize} class template chains calls to its specializations, resulting in a completely unrolled sequence of 
calls to \texttt{\_plus::apply} each using the best available vectorization (except for the remainder if $N \mod \texttt{STEP} \neq 0$, where the next best is used).

\begin{lstlisting}
/**
 * recursively call binary operator kernels 
 */
template <class Op, int N>
struct _vectorize {
    template <class vec>
    static inline const void apply(vec & left, const vec & right, const int start = 0) {
        _vectorize<Op, STEP>::apply(left, right, start);
        _vectorize<Op, N - STEP>::apply(left, right, start + STEP);
    }
};

template <class Op>
struct _vectorize<Op, 0>
{
    template <class vec>
    static inline const void apply(vec & left, const vec & right, const int start = 0) { }
};

template <class Op>
struct _vectorize<Op, 1>
{
    template <class vec>
    static inline const void apply(vec & left, const vec & right, const int start = 0) {
        Op::apply(left, right, start);
    }
};

#ifdef __SSE2__

template <class Op>
struct _vectorize<Op, 2>
{
    template <class vec>
    static inline const void apply(vec & left, const vec & right, const int start = 0) {
        double * l = left.data();
        const double * r = right.data();
        __m128d mml, mmr, mms;
        mml = _mm_load_pd(l + start);
        mmr = _mm_load_pd(r + start);
        mms = Op::apply(mml, mmr);
        _mm_store_pd(l + start, mms);
    }
};

#endif

#ifdef __AVX__

template <class Op>
struct _vectorize<Op, 3>
{
    template <class vec>
    static inline const void apply(vec & left, const vec & right, const int start = 0) {
        _vectorize<Op, 2>::apply(left, right, start);
        _vectorize<Op, 1>::apply(left, right, start + 2);
    }
};

template <class Op>
struct _vectorize<Op, 4>
{
    template <class vec>
    static inline const void apply(vec & left, const vec & right, const int start = 0) {
        double * l = left.data();
        const double * r = right.data();
        __m256d mml, mmr, mms;
        mml = _mm256_load_pd(l + start);
        mmr = _mm256_load_pd(r + start);
        mms = Op::apply(mml, mmr);
        _mm256_store_pd(l + start, mms);
    }
};

#endif
\end{lstlisting}  

And finally the full definition for \texttt{operator+=}:

\begin{lstlisting}
/**
 * operator overload
 */
template <int N>
inline const tinyvector<double, N, INTRIN_OPT> & operator+= (tinyvector<double, N, INTRIN_OPT> & left, const tinyvector<double, N, INTRIN_OPT> & right) {
    _vectorize<_plus, N>::apply(left, right);
    return left;
}
\end{lstlisting}

\subsection*{Benchmarking}
The benchmark was written to be close to the use case, a class holding a \texttt{std::vector} of \texttt{tinyvector}s 
with a function that loops through the vector applying the operation to be benchmarked to every item in turn.
The runtime was measured in clock cycles using RDTSC.

The benchmarks were run on the EULER cluster.

\section*{Results}
The Speedup of the optimized versus the naive default implementation for each of the four basic math operators was measured and plotted (\cref{fig:su-pl,fig:su-mi,fig:su-mu,fig:su-di}).

\begin{figure}
    \begin{subfigure}{\textwidth}
        \includegraphics[width=0.8\textwidth]{results/su_plus.pdf}
        \caption{addition operator.}
        \label{fig:su-pl} 
    \end{subfigure}

    \begin{subfigure}{\textwidth}
        \includegraphics[width=0.8\textwidth]{results/su_minus.pdf}
        \caption{subtraction operator.}
        \label{fig:su-mi}
    \end{subfigure}
    \caption{Speedup of optimized vs. naive version for plus and minus operators for some tinyvector dimensionalities. \texttt{\# of tinyvectors} is the size of the \texttt{std::vector} containing the tinyvectors.}
\end{figure}

\begin{figure}
    \centering
    \begin{subfigure}{\textwidth}
        \includegraphics[width=0.8\textwidth]{results/su_multiply.pdf}
        \caption{(element wise) multiplication operator.}
        \label{fig:su-mu}
    \end{subfigure}

    \begin{subfigure}{\textwidth}
        \includegraphics[width=0.8\textwidth]{results/su_divide.pdf}
        \caption{(element wise) division operator.}
        \label{fig:su-di}
    \end{subfigure}
    \caption{Speedup of optimized vs. naive version for multiplication and division operators for some tinyvector dimensionalities. \texttt{\# of tinyvectors} is the size of the \texttt{std::vector} containing the tinyvectors.}
\end{figure}

\end{document}
